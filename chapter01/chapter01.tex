\chapter{Contr\^ole des configurations de c\^ables}

Nous proposons dans ce chapitre une m\'ethode permettant de contr\^oler les
c\^ables qui seront en tension le long d'une trajectoire donn\'ee. Dans une
premi\`ere section, nous expliquerons les circonstances qui nous ont
conduits \`a d\'evelopper une m\'ethode de contr\^ole, puis nous introduirons
les concepts n\'ecessaires \`a son d\'eveloppement. La seconde section de ce
chapitre sera consacr\'ee \`a l'\'elaboration de crit\`eres qui nous serviront
\`a s\'electionner les configurations optimisant la stabilit\'e et la
pr\'ecision d'une trajectoire. Enfin, la troisi\`eme et derni\`ere section de
ce chapitre introduira dans un premier temps l'analyse par intervalle dont nous
nous serons servis afin de garantir notre d\'emarche ; puis nous pr\'esenterons
les principes de l'algorithme d\'evelopp\'e et concluerons par l'exposition des
r\'esultats th\'eoriques qui seront valid\'es dans le chapitre consacr\'e aux
simulations et exp\'eri\-mentations.

\section{Notions introductives}

\subsection{Contexte}

Nous avons vu que l'une des sp\'ecificit\'es des manipulateurs parall\`eles \`a
c\^ables est que -- pour une pose donn\'ee -- l'\'equilibre statique peut
\^etre obtenu sans que l'ensemble des c\^ables soient en tension. De plus, il
est possible que dans certaines configurations, il est impossible pour toute
pose que l'ensemble des c\^ables aient une tension strictement positive : ainsi,
pour une configuration $N-1$, avec $N> 3$ et des c\^ables non-\'elastiques, il
est prouv\'e que 3 c\^ables au maximum auront une tension strictement positive,
les $N-3$ autres \'etant d\'etendus \cite{merlet2012}.

La proc\'edure classique consiste \`a donner aux c\^ables d\'etendus une
longueur $\rho_i$ \'egale \`a la distance $||A_iB_i||$. Or, le r\'esultat
peut-\^etre le plus important ici est donn\'e dans \cite{merlet2014} : bien
qu'il s'agisse d'un travail effectu\'e dans le cas de c\^ables \'elastiques,
les r\'esultats restent valables dans la situation in\'elastique, et montrent
qu'il est impossible en pratique de pr\'evoir les modifications du caract\`ere
strictement positif ou nul de la tension pour l'ensemble des c\^ables lors d'une
trajectoire donn\'ee. Ainsi, une modification infime de la longueur du c\^able
en tension nulle dont nous avons r\'egl\'e la longueur de mani\`ere \`a ce que
$\rho_i = A_iB_i$ pourra changer compl\`etement la nature de la tension dans
les autres c\^ables et surtout modifier la pose de l'organe terminal.

La cons\'equence de ces ph\'enom\`enes m\'ecaniques est double : d'une part
nous pouvons nous retrouver dans des configurations sous-contraintes, nous
perdons alors le contr\^ole d'un ou plusieurs degr\'es de libert\'es et la
trajectoire n'est plus stable ; d'autre part les modifications soudaines des
tensions perturbent les mouvements de la plate-forme et la pr\'ecision du
positionnement est affect\'ee.

En dehors des questions de s\'ecurit\'es soulev\'ees par ces situations, nous
avons plusieurs fois observ\'e que ces changements subits engendraient des
mouvements ind\'esirables au niveau de l'organe terminal. Principalement,
puisque les degr\'es de libert\'es en rotation ne sont pas contraints sur notre
prototype, des mouvements pendulaires ont parfois \'et\'e g\'en\'er\'es,
conduisant \`a la perte de la cible dans l'image, et donc l'\'echec de
l'asservissement.

Il nous a donc paru n\'ecessaire dans un premier temps de mettre en place une
strat\'egie permettant de garantir la stabilit\'e des d\'eplacements et la
pr\'ecision des positionnements toutes les fois o\`u cela \'etait possible, et
de pouvoir identifier les situations pour lesquelles il n'y a d'autre solution
que d'adopter un mode de fonctionnement d\'egrad\'e.

\subsection{Les configurations de c\^ables}

Nous appelons {\it configuration de c\^ables}, ou $CC$, la donn\'ee de
l'ensemble des c\^ables qui sont en tension strictement positive. Ainsi, pour
un robot \`a $n$ c\^ables pour $d$ degr\'es de libert\'e, nous noterons
$CC_{k_1, k_2, \dots k_m}$ la configuration pour laquelle les c\^ables $k_1,
k_2, \dots k_m$ sont en tension strictement positive et -- si $m < n$ --  les
c\^ables $k_{m+1}, \dots k_n$ sont d\'etendus. Si la configuration est telle
qu'elle permet avec les $m$ c\^ables en tension de contr\^oler les $d$ de
degr\'es de libert\'es, nous la noterons $\overline{CC}_{k_1, k_2, \dots k_m}$.

Si l'ensemble $k_1, \dots k_p$ est strictement inclus dans l'ensemble $k_1,
\dots k_m$, on pourra \'ecrire $CC_{k_1, \dots k_p} < CC_{k_1, \dots
k_m}$.

Prenons l'exemple d'un robot $4-1$ dont les c\^ables $0, 1, 2, 3$ sont
dispos\'es de mani\`ere \`a ce qu'aucun triplet $A_iA_jA_k$ de points d'attaches
ne soit align\'e. Les configurations possibles pour un tel robot sont :
$\overline{CC}_{012}, \overline{CC}_{013}, \overline{CC}_{023},
\overline{CC}_{123}$, $CC_{01}, CC_{02}, CC_{03}, CC_{12}, CC_{13}, CC_{23},
CC_{0}, CC_{1}, CC_{2}, CC_{3}$. On notera ici que la configuration
${CC}_{0123}$ n'a pas de sens sur ce robot particulier.

Nous appelons {\it coordonn\'ees situ\'ees} ${}^s{\bf P}$ d'un point
${\bf P}$ la donn\'ee de ses param\`etres de pose ${\bf X}$ et de la
configuration de c\^ables dans laquelle il se trouve :

$${}^s{\bf P} := ({\bf X} ; CC_{k_1, \dots k_m})$$

L'ensemble des points $\bf P$ tels qu'il existe ${}^s {\bf P} :=
({\bf X} ; CC_{k_1, \dots k_m})$ pour $CC_{k_1, \dots k_m}$ fix\'e est appel\'e
l'{\it espace l\'egal} de $CC_{k_1, \dots k_m}$, not\'e $El(CC_{k_1, \dots
k_m})$. Nous noterons de plus $\overline{El}(CC_{k_1, \dots
k_m})$ la fermeture de ${El}(CC_{k_1, \dots k_m})$, \`a savoir l'ensemble :
$$\overline{El}(CC_{k_1, \dots k_m}) := \bigcup
El(CC_{k_1, \dots k_p}), CC_{k_1, \dots k_p} < CC_{k_1, \dots k_m}$$

De la m\^eme mani\`ere, pour un m\^eme point $\bf P$, l'ensemble $Sl$
correspond \`a l'ensemble des configurations de c\^ables $CC_{k_1, \dots k_m}$
telles qu'il existe ${}^s{\bf P} \in El(CC_{k_1, \dots k_m})$. Ici,
$\overline{Sl}$ repr\'esentera l'ensemble des $\overline{CC}_{k_1, \dots k_m}$
pour lesquelles il existe ${}^s{\bf P} \in El(\overline{CC}_{k_1, \dots k_m})$.

Soit maintenant $\mathcal O({\bf S}, {\bf G})$ l'ensemble des points ${\bf
P}_i$ qui constituent une trajectoire allant du point ${\bf S}$ au point ${\bf
G}$. Nous noterons $\mathcal O({\bf S}, {\bf G}) \vartriangleleft \subset
El(CC_{k_1, \dots k_m})$ la situation pour laquelle $\forall {\bf P_i} \in
\mathcal O({\bf S}, {\bf G}), \exists {}^s{\bf P}_i \in El(CC_{k_1, \dots
k_m})$, soit que la trajectoire appartient int\'egralement \`a l'espace l\'egal
de $CC_{k_1, \dots k_m}$.

Lorsque nous n'avons pas cette relation, nous avons besoin de d\'efinir des
r\'egions de transferts. Soit deux configurations de c\^ables
$CC_{k_1, \dots k_m}$ et $CC_{k_1, \dots k_p}$, que pour un instant nous
noterons respectivement $CC_i$ et $CC_j$. Nous d\'efinissons l'ensemble
$\overset{\circ}{Tr_{ij}} = El(CC_i) \cap El(CC_j)$ comme l'ensemble des points
${\bf P}$ pour lesquels il existe ${}^sP_i \in El(CC_i)$ et ${}^sP_j \in
El(CC_j)$. Un changement de coordonn\'ees situ\'ees de ${}^sP_i$ vers ${}^sP_j$
sera dans ce cas appel\'e un {\it transfert simple}. Lorsque
$\overset{\circ}{Tr_{ij}} = \emptyset$, mais qu'il existe, pour un point, ${\bf
P}$ ${}^sP_i \in \overline{El}(CC_i)$ et ${}^sP_j \in \overline{El}(CC_j)$, nous
appelerons $\partial Tr_{ij}$ l'ensemble des points de {\it transfert marginal}
entre les configurations $CC_i$ et $CC_j$. L'espace de tranfert total $Tr_{ij}$
est alors la r\'eunion des espaces de transferts simple et marginal
correspondants.


\subsection{Robots d\'eriv\'es et robot int\'egral}

\section{Crit\`eres d'optimisation et d'\'evaluation}

\subsection{Am\'elioration de la stabilit\'e}

\subsubsection{Etat de l'art}

\subsubsection{Crit\`ere}

\subsubsection{Illustration}

\subsection{Am\'elioration de la pr\'ecision}

\subsubsection{Etat de l'art}

\subsubsection{Crit\`ere}

\subsubsection{Illustration}

\section{Algorithme de s\'election de configurations de c\^ables}

\subsection{L'analyse par intervalles}

\subsection{Algorithme}

\subsection{R\'esultats th\'eoriques}
