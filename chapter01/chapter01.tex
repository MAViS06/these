\chapter{Contr\^ole des configurations de c\^ables}

Nous proposons dans ce chapitre une m\'ethode permettant de contr\^oler les
c\^ables qui seront en tension le long d'une trajectoire donn\'ee. Dans une
premi\`ere section, nous expliquerons les circonstances qui nous ont
conduits \`a d\'evelopper une m\'ethode de contr\^ole, puis nous introduirons
les concepts n\'ecessaires \`a son d\'eveloppement. La seconde section de ce
chapitre sera consacr\'ee \`a l'\'elaboration de crit\`eres qui nous serviront
\`a s\'electionner les configurations optimisant la stabilit\'e et la
pr\'ecision d'une trajectoire. Enfin, la troisi\`eme et derni\`ere section de
ce chapitre introduira dans un premier temps l'analyse par intervalle dont nous
nous serons servis afin de garantir notre d\'emarche ; puis nous pr\'esenterons
les principes de l'algorithme d\'evelopp\'e et concluerons par l'exposition des
r\'esultats th\'eoriques qui seront valid\'es dans le chapitre consacr\'e aux
simulations et exp\'eri\-mentations.

\section{Notions introductives}

\subsection{Contexte}

Nous avons vu que l'une des sp\'ecificit\'es des manipulateurs parall\`eles \`a
c\^ables est que -- pour une pose donn\'ee -- l'\'equilibre statique peut
\^etre obtenu sans que l'ensemble des c\^ables soient en tension. De plus, il
est possible que dans certaines configurations, il est impossible pour toute
pose que l'ensemble des c\^ables aient une tension strictement positive : ainsi,
pour une configuration $N-1$, avec $N> 3$ et des c\^ables non-\'elastiques, il
est prouv\'e que 3 c\^ables au maximum auront une tension strictement positive,
les $N-3$ autres \'etant d\'etendus \cite{merlet2012}.

La proc\'edure classique consiste \`a donner aux c\^ables d\'etendus une
longueur $\rho_i$ \'egale \`a la distance $||A_iB_i||$. Or, le r\'esultat
peut-\^etre le plus important ici est donn\'e dans \cite{merlet2014} : bien
qu'il s'agisse d'un travail effectu\'e dans le cas de c\^ables \'elastiques,
les r\'esultats restent valables dans la situation in\'elastique, et montrent
qu'il est impossible en pratique de pr\'evoir les modifications du caract\`ere
strictement positif ou nul de la tension pour l'ensemble des c\^ables lors d'une
trajectoire donn\'ee. Ainsi, une modification infime de la longueur du c\^able
en tension nulle dont nous avons r\'egl\'e la longueur de mani\`ere \`a ce que
$\rho_i = A_iB_i$ pourra changer compl\`etement la nature de la tension dans
les autres c\^ables et surtout modifier la pose de l'organe terminal.

La cons\'equence de ces ph\'enom\`enes m\'ecaniques est double : d'une part
nous pouvons nous retrouver dans des configurations sous-contraintes, nous
perdons alors le contr\^ole d'un ou plusieurs degr\'es de libert\'es et la
trajectoire n'est plus stable ; d'autre part les modifications soudaines des
tensions perturbent les mouvements de la plate-forme et la pr\'ecision du
positionnement est affect\'ee.

En dehors des questions de s\'ecurit\'es soulev\'ees par ces situations, nous
avons plusieurs fois observ\'e que ces changements subits engendraient des
mouvements ind\'esirables au niveau de l'organe terminal. Principalement,
puisque les degr\'es de libert\'es en rotation ne sont pas contraints sur notre
prototype, des mouvements pendulaires ont parfois \'et\'e g\'en\'er\'es,
conduisant \`a la perte de la cible dans l'image, et donc l'\'echec de
l'asservissement.

Il nous a donc paru n\'ecessaire dans un premier temps de mettre en place une
strat\'egie permettant de garantir la stabilit\'e des d\'eplacements et la
pr\'ecision des positionnements toutes les fois o\`u cela \'etait possible, et
de pouvoir identifier les situations pour lesquelles il n'y a d'autre solution
que d'adopter un mode de fonctionnement d\'egrad\'e.

\subsection{Les configurations de c\^ables}

Nous appelons {\it configuration de c\^ables}, ou $CC$, la donn\'ee de
l'ensemble des c\^ables qui sont en tension strictement positive. Ainsi, pour
un robot \`a $n$ c\^ables pour $d$ degr\'es de libert\'e, nous noterons
$CC_{k_1, k_2, \dots k_m}$ la configuration pour laquelle les c\^ables $k_1,
k_2, \dots k_m$ sont en tension strictement positive et -- si $m < n$ --  les
c\^ables $k_{m+1}, \dots k_n$ sont d\'etendus. Si la configuration est telle
qu'elle permet avec les $m$ c\^ables en tension de contr\^oler les $d$ de
degr\'es de libert\'es, nous la noterons $\overline{CC}_{k_1, k_2, \dots k_m}$.

Si l'ensemble $k_1, \dots k_p$ est strictement inclus dans l'ensemble $k_1,
\dots k_m$, on pourra \'ecrire $CC_{k_1, \dots k_p} < CC_{k_1, \dots
k_m}$.

Prenons l'exemple d'un robot $4-1$ dont les c\^ables $0, 1, 2, 3$ sont
dispos\'es de mani\`ere \`a ce qu'aucun triplet $A_iA_jA_k$ de points d'attaches
ne soit align\'e. Les configurations possibles pour un tel robot sont :
$\overline{CC}_{012}, \overline{CC}_{013}, \overline{CC}_{023},
\overline{CC}_{123}$, $CC_{01}, CC_{02}, CC_{03}, CC_{12}, CC_{13}, CC_{23},
CC_{0}, CC_{1}, CC_{2}, CC_{3}$. On notera ici que la configuration
${CC}_{0123}$ n'a pas de sens sur ce robot particulier.

Nous appelons {\it coordonn\'ees situ\'ees} ${}^s{\bf P}$ d'un point
${\bf P}$ la donn\'ee de ses param\`etres de pose ${\bf X}$ et de la
configuration de c\^ables dans laquelle il se trouve :

$${}^s{\bf P} := ({\bf X} ; CC_{k_1, \dots k_m})$$

L'ensemble des points $\bf P$ tels qu'il existe ${}^s {\bf P} :=
({\bf X} ; CC_{k_1, \dots k_m})$ pour $CC_{k_1, \dots k_m}$ fix\'e est appel\'e
l'{\it espace l\'egal} de $CC_{k_1, \dots k_m}$, not\'e $El(CC_{k_1, \dots
k_m})$. Nous noterons de plus $\overline{El}(CC_{k_1, \dots
k_m})$ la fermeture de ${El}(CC_{k_1, \dots k_m})$, \`a savoir l'ensemble :
$$\overline{El}(CC_{k_1, \dots k_m}) := \bigcup
El(CC_{k_1, \dots k_p}), CC_{k_1, \dots k_p} < CC_{k_1, \dots k_m}$$

De la m\^eme mani\`ere, pour un m\^eme point $\bf P$, l'ensemble $Sl$
correspond \`a l'ensemble des configurations de c\^ables $CC_{k_1, \dots k_m}$
telles qu'il existe ${}^s{\bf P} \in El(CC_{k_1, \dots k_m})$. Ici,
$\overline{Sl}$ repr\'esentera l'ensemble des $\overline{CC}_{k_1, \dots k_m}$
pour lesquelles il existe ${}^s{\bf P} \in El(\overline{CC}_{k_1, \dots k_m})$.

Soit maintenant $\mathcal O({\bf S}, {\bf G})$ l'ensemble des points ${\bf
P}_i$ qui constituent une trajectoire allant du point ${\bf S}$ au point ${\bf
G}$. Nous noterons $\mathcal O({\bf S}, {\bf G}) \vartriangleleft
El(CC_{k_1, \dots k_m})$ la situation pour laquelle $\forall {\bf P_i} \in
\mathcal O({\bf S}, {\bf G}), \exists {}^s{\bf P}_i \in El(CC_{k_1, \dots
k_m})$, soit que la trajectoire appartient int\'egralement \`a l'espace l\'egal
de $CC_{k_1, \dots k_m}$.

Lorsque nous n'avons pas cette relation, nous avons besoin de d\'efinir des
r\'egions de transferts. Soit deux configurations de c\^ables
$CC_{k_1, \dots k_m}$ et $CC_{k_1, \dots k_p}$, que pour un instant nous
noterons respectivement $CC_i$ et $CC_j$. Nous d\'efinissons l'ensemble
$\overset{\circ}{Tr_{ij}} = El(CC_i) \cap El(CC_j)$ comme l'ensemble des points
${\bf P}$ pour lesquels il existe ${}^sP_i \in El(CC_i)$ et ${}^sP_j \in
El(CC_j)$. Un changement de coordonn\'ees situ\'ees de ${}^sP_i$ vers ${}^sP_j$
sera dans ce cas appel\'e un {\it transfert simple}. Lorsque
$\overset{\circ}{Tr_{ij}} = \emptyset$, mais qu'il existe, pour un point ${\bf
P}$, ${}^sP_i \in \overline{El}(CC_i)$ et ${}^sP_j \in \overline{El}(CC_j)$, nous
appelerons $\partial Tr_{ij}$ l'ensemble des points de {\it transfert marginal}
entre les configurations $CC_i$ et $CC_j$. L'espace de transfert total $Tr_{ij}$
est alors la r\'eunion des espaces de transferts simple et marginal
correspondants.

Supposons par exemple que ${\bf S} \in El(CC_1)$, ${\bf G} \in El(CC_2)$ et qu'il existe un point ${\bf M}$ tel que $\mathcal O({\bf S}, {\bf M}) \vartriangleleft
El(CC_1)$ et $\mathcal O({\bf M}, {\bf G}) \vartriangleleft El(CC_2)$, alors on pourra écrire $\mathcal O({\bf S}, {\bf G}) \vartriangleleft El(CC_1) \cup \overset{\circ}{Tr_{12}} \cup El(CC_2)$. De la même manière, si un point ${\bf M}$ est tel que $\mathcal O({\bf S}, {\bf M}) \vartriangleleft
\overline{El}(CC_1)$ et $\mathcal O({\bf M}, {\bf G}) \vartriangleleft \overline{El}(CC_2)$, alors nous pourrons écrire $\mathcal O({\bf S}, {\bf G}) \vartriangleleft El(CC_1) \cup \partial Tr_{12} \cup El(CC_2)$. Nous appellerons {\it trajet} l'ensemble des configurations de c\^ables et régions de transferts nécessaires au parcours d'une trajectoire. On distinguera les situations suivantes :
\begin{itemize}
 \item {\it trajet trivial} : l'intégralité de la trajectoire peut être réalisée à partir d'une seule configuration de câbles.
 \item {\it trajet simple} : la trajectoire requiert au moins un transfert ; tous les transferts sont simples.
 \item {\it trajet partiellement marginal} : la trajectoire requiert au moins un transfert ; il existe au moins un transfert simple et un transfert marginal.
 \item {\it trajet marginal} : la trajectoire requiert au moins un transfert ; tous les transferts sont marginaux.
\end{itemize}

Lorsque plusieurs trajets existent, nous essaierons alors de définir des relations d'ordre nous permettant de sélectionner le meilleur trajet possible en fonction d'un ou plusieurs critères préalablement définis : ce sera l'objet de la seconde section de ce chapitre.

\subsection{Robots d\'eriv\'es et robot int\'egral}

Soit un robot $4-3-1$, contrôlé par $4$ câbles dont $3$ (les câbles $i, j, k$) sont attachés au même point ${\bf B}_1$ sur l'organe terminal, et un quatrième (câble $l$) relié à celle-ci en un point ${\bf B}_2 \neq {\bf B}_1$. Il a été montré dans \cite{merlet2013-431} que pour une pose donnée, nous pouvions nous retrouver dans des configurations telles que l'un des câbles $i, j$ ou $k$ est détendu -- auquel cas l'auteur parle de configuration $3-2-1$) -- ou encore le câble $l$ -- il s'agira alors d'une configuration $3-1$). Ces deux situations présentent des différences non négligeables du point de vue des propriétés du robot, tout autant au niveau du contrôle que des différents modèles géométriques et cinématiques. La question se pose alors de savoir s'il est plus pertinent d'adopter une démarche {\it top-down} consistant à rechercher les propriétés du robot $4-3-1$ et les décliner ensuite selon les situations, ou au contraire une démarche {\it bottom-up} qui partirait des propriétés spécifiques à chaque possibilité de configuration pour ensuite construire un fonctionnement global.

Un second argument en faveur de cette interrogation est soulevé dans \cite{merlet2012}. Pour rappel, l'équilibre statique s'exprime par la formule suivante : $\mathcal {\bf F} = {\bf J}^{-T} {\bf \tau}$. Or, afin de déterminer les tensions dans les différents câbles, il faut inverser cette relation. Lorsque la matrice jacobienne n'est pas inversible ({\it a fortiori} lorsqu'elle n'est pas carrée), l'auteur montre que l'utilisation de la matrice pseudo-inverse n'est pas appropriée et donne des résultats qui peuvent être faux. On peut se demander dès lors si l'expression de cette première jacobienne est la plus pertinente, s'il n'existe pas une autre approche de la modélisation du robot qui en permettrait une meilleure intuition et utilisation.

Soit $A^{n, m}$ un manipulateur parall\`ele \`a $m$ c\^ables d\'evelopp\'e pour 
contr\^oler au plus $n$ degr\'es de libert\'es, que nous appellerons {\it 
robot maximal}. Le couple ${n, m}$ repr\'esente la {\it signature} du robot et 
sera ici appel\'ee sa {\it signature maximale}.

Associ\'e à un espace de travail $W$, nous dirons qu'il est
\begin{itemize}
 \item {r\'ealisable} s'il existe une pose dans $W$ telle que l'ensemble des 
c\^ables sont en tension
  \item {\it concr\^et} si pour toute pose de $W$ l'ensemble des c\^ables sont 
en tension
  \item {\it abstrait} si il n'existe aucune pose dans $W$ telle que l'ensemble 
des c\^ables sont en tension
\end{itemize}

Nous d\'efinissons une première op\'eration appel\'ee {\it concr\^etisation} 
consistant pour un robot r\'ealisable \`a lui associer le sous-espace de $W_i 
\subset W$ pour lequel il pourra \^etre dit concr\^et. 

Nous \'etablissons ensuite la liste des sous-configurations de c\^ables 
possibles. Nous appelons alors {\it sous-robot} $A_i$ associ\'e \`a la 
configuration de c\^ables $CC_i$ le robot virtuel d\'efini \`a partir des seuls 
c\^ables correspondant \`a sa configuration de c\^ables. Un tel sous-robot 
pourra \`a son tour \^etre abstrait, r\'ealisable ou concr\^et.

Lorsque pour deux configurations de c\^ables $CC_i$ et $CC_j$ nous avons $CC_i< 
CC_j$, alors le sous-robot $A_i$ associ\'e \`a $CC_i$ sera consid\'er\'e comme 
{\it d\'eriv\'e} de $A_j$. S'il existe un ensemble $CC_\alpha$ et un robot 
$A_j$ associ\'e \`a la configuration de c\^ables $CC_j$ tels que $\forall CC_i 
\in CC_alpha, CC_i < CC_j$ et $\forall CC_k \notin CC_alpha, CC_i \nless 
CC_j$, alors on dira de $A_j$ qu'il est un {\it robot int\'egral} des robots 
associ\'es aux configurations de c\^ables $CC_i \in CC_\alpha$.

Pour chaque robot $A_i$ d\'eriv\'e ou int\'egral, nous pouvons, lorsqu'il est 
r\'ealisable, lui associer son sous-espace concr\^etis\'e $W_i$. Notons que 
si $A_i$ d\'erive de $A_j$, cela ne signifie pas pour autant que $W_i 
\subset W_j$. Nous adjoindrons \'egalement \`a tout robot $A_i$ une signature 
$(p,q)$, $q$ repr\'esentant le nombre de degr\'es de libert\'es qu'il est 
possible de contr\^oler gr\^ace \`a sa configuration de $p$ c\^ables.

Nous partons donc du robot maximal pour en d\'eduire les diff\'erents 
sous-robots possibles. Parmi ceux-ci, nous distinguons ceux qui entretiennent 
des relations de d\'erivation et int\'egration tels que nous les avons 
d\'efinies. Puis il nous faut pour chaque sous-robot \'etudier ses 
propri\'et\'es propres et s'interroger sur la mani\`ere dont elles \'evoluent 
entre robots d\'eriv\'es et int\'egraux. Il doit alors \^etre possible en 
fonction d'une t\^ache prescrite d'obtenir une relation d'ordre entre les 
sous-robots et de choisir celui qui nous semble le plus appropri\'e \`a la 
r\'ealisation de cette t\^ache.

Parmi les propri\'et\'es qui peuvent \^etre d\'eduites par relations de 
d\'erivations et int\'egrations, la d\'etermination de la Jacobienne nous 
int\'eresse particuli\`erement dans le cadre de ce travail. Nous d\'efinissons 
d\`es lors les op\'erateurs suivantes :

\begin{itemize}
 \item {\it d\'erivateur} : soient $A_i$ et $A_j$ tels que le premier d\'erive 
du second, et ${\bf J}^{-1}_i$, ${\bf J}^{-1}_j$ leurs matrices jacobiennes 
inverses respectives ; on appelle {\it d\'erivateur} la matrice ${\bf D}$ telle 
que $ {\bf D} {\bf J}^{-1}_j = {\bf J}^{-1}_i$.
 \item {\it int\'egrateurs} : soit $A_j$ un robot int\'egral des robots 
$A_{i_1}, \dots, A_{i_p}$ ; on appelle {\it int\'egrateurs} les matrices 
${\bf I}_1, \dots {\bf I}_p$ tels que $\sum_{k=1}^p {\bf I}_k {\bf J}^{-1}_k = 
{\bf J}^{-1}_j$, avec ${\bf J}^{-1}_k$ les matrices jacobiennes inverses 
respectives des robots $A_{i_k}$.
\end{itemize}


\subsection{Illustration avec le robot {\tt Marionet-Assist}}

\vskip -100pts

\begin{figure}[!ht]
  \centering
    \def\svgwidth{.65\linewidth}
  \input{./chapter01/figures/CC_41.pdf_tex}
    \caption{\footnotesize{Les diff\'erentes configurations de c\^ables du 
{\tt Marionet-Assist} : les fl\`eches noires repr\'esentent les inclusions et 
les lignes en pointill\'es bleus indiquent une r\'egion de transfert simple.}}
\label{chap01:fig0}
\end{figure}

Equip\'e de $4$ c\^ables reli\'es en un m\^eme point \`a l'organe 
terminal (index\'es dans le sens trigonom\'etrique en partant du coin 
bas-gauche), les diff\'erentes confi\-gurations de c\^ables pouvant \^etre 
obtenues \`a partir de {\tt Marionet-Assist} sont :
\begin{itemize}
 \item $4 CC$ \`a $1$ c\^able : $CC_0, CC_1, CC_2, CC_3$
  \item $6 CC$ \`a $2$ c\^ables : $CC_{01}, CC_{12}, CC_{23}, CC_{30}, CC_{02}, 
CC_{13}$, les deux derni\`eres pr\'esentant la caract\'eristique que 
$El(CC_{02}) \cap El(CC_{13}) \neq \emptyset$ (il existe donc un transfert 
simple \`a 2 degr\'es de libert\'e entre ces deux configurations).
  \item $4 CC$ \`a $3$ c\^ables : $\overline{CC}_{012}, \overline{CC}_{013}, 
\overline{CC}_{023}, \overline{CC}_{123}$. Il existe de plus des r\'egions de 
transferts simples entre plusieurs de ces configurations de c\^ables.
\end{itemize}

Ces relations sont reprises dans la Fig.\ref{chap01:fig0}. Le graphe ainsi 
obtenu permet par exemple de d\'efinir les {\it trajets} possibles pour une 
s\'equence donn\'ee, et de calculer le {\it trajet} optimisant un crit\`ere en 
attribuant les poids correspondants aux noeuds et fl\`eches du graphe. On voit 
par exemple que pour passer de $\overline{CC}_{013}$ \`a $\overline{CC}_{123}$, 
il est possible d'utiliser une {\it transition marginale} via $CC_{13}$, ou 
encore deux trajets \`a une transition simple, l'un passant par 
$\overline{CC}_{012}$ et l'autre par $\overline{CC}_{023}$, ou bien encore 
d'autres combinaisons marginales et partiellement marginales.

Le robot maximal et ses sous-robots sont repr\'esent\'es dans la 
Fig.\ref{chap01:fig1}, ainsi que leurs relations de d\'erivation et 
d'int\'egration. Dans le cas du robot {\tt Marionet-Assist}, le robot maximal 
est abstrait, et il existe $4$ robots r\'ealisables dans $A^{3,3}$, $6$ dans 
$A^{2,2}$ et quatre dans $A^{1,1}$. Pour chacun des robots r\'ealisables, 
l'espace de concr\^etisation correspond \`a l'ouvert obtenu par projection 
verticale des droites reliant les points d'attaches correspondant aux 
configurations de c\^ables associ\'ees. Ainsi, l'espace de concr\^etisation de 
$A^{3,3}_{012}$ est obtenu par projection verticale du triangle issu des 
droites ${\bf A}_0{\bf A}_1$, ${\bf A}_1{\bf A}_2$ et ${\bf A}_0{\bf A}_2$, \`a 
l'exception de la projection verticale de ces m\^emes droites. De la m\^eme 
mani\`ere, l'espace de concr\^etisation du robot $A^{2,2}_{0,1}$ est obtenu par 
projection verticale de la droite ${\bf A}_0{\bf A}_1$, \`a l'exception de la 
projection verticale des points ${\bf A}_0$ et ${\bf A}_1$.

\begin{figure}[htp]
  \centering
    \def\svgwidth{.65\linewidth}
  \input{./chapter01/figures/rbt41.pdf_tex}
    \caption{\footnotesize{Schéma de d\'erivation/int\'egration du robot {\tt 
Marionet-Assist} : s'il existe une fl\^eche allant de $CC_i$ \`a $CC_j$, 
alors $CC_i$ d\'erive de $CC_j$ ; $CC_j$ sera le robot int\'egral de l'ensemble 
des robots $CC_k$ qui en d\'erivent.}}
\label{chap01:fig1}
\end{figure}

Soit ${\bf B}$ le point d'attache des c\^ables \`a la plate-forme, ${\bf 
A}_0$, ${\bf A}_1$, ${\bf A}_2$ et ${\bf A}_3$ les points de sorties 
respectifs des $4$ c\^ables, et $\rho_0$, $\rho_1$, $\rho_2$ et $\rho_3$ les 
longueurs de c\^ables associ\'ees. La jacobienne inverse du robot est donn\'ee 
par la matrice dont les lignes sont les vecteurs $\frac {{\bf A}_i{\bf 
B}}{\rho_i}$. La {\it jacobienne inverse maximale} est donc :
\begin{equation}
{\bf J}^{-1} = 
\begin{bmatrix}
\frac {{\bf A}_{0_x}{\bf B}_x} {\rho_0} & \frac {{\bf A}_{0_y}{\bf B}_y} 
{\rho_0} & \frac {{\bf A}_{0_z}{\bf B}_z} {\rho_0} \\
\frac {{\bf A}_{1_x}{\bf B}_x} {\rho_1} & \frac {{\bf A}_{1_y}{\bf B}_y} 
{\rho_1} & \frac {{\bf A}_{1_z}{\bf B}_z} {\rho_1} \\
\frac {{\bf A}_{2_x}{\bf B}_x} {\rho_2} & \frac {{\bf A}_{2_y}{\bf B}_y} 
{\rho_2} & \frac {{\bf A}_{2_z}{\bf B}_z} {\rho_2} \\
\frac {{\bf A}_{3_x}{\bf B}_x} {\rho_3} & \frac {{\bf A}_{3_y}{\bf B}_y} 
{\rho_3} & \frac {{\bf A}_{3_z}{\bf B}_z} {\rho_3} \\
\end{bmatrix}
\label{chap01:eq01}
\end{equation}

Le d\'erivateur permettant de d\'eduire la jacobienne inverse associ\'ee au 
robot $A^{3,3}_{0, 1, 2}$ \`a partir de la jacobienne inverse maximale 
sera la matrice :
\begin{equation}
{\bf D}_{012/3} = 
\begin{bmatrix}
1 & 0 & 0 & 0\\
0 & 1 & 0 & 0\\
0 & 0 & 1 & 0
\end{bmatrix}
\label{chap01:eq02}
\end{equation}

Le d\'erivateur permettant de d\'eduire la jacobienne inverse du robot 
$A^{2,2}_{0, 1}$ \`a partir de $J^{-1}_{012}$ sera :
\begin{equation}
{\bf D}_{01/2} = 
\begin{bmatrix}
1 & 0 & 0\\
0 & 1 & 0
\end{bmatrix}
\label{chap01:eq03}
\end{equation}

Les int\'egrateurs permettant par exemple de d\'eduire $A^{3,3}_{0, 1, 2}$ \`a 
partir de $A^{2,2}_{0, 1}$, $A^{2,2}_{0, 2}$ et $A^{2,2}_{1,2}$ seront 
respectivement :
\begin{equation}
{\bf I}_{01+2} = 1/2
\begin{bmatrix}
1 & 0 \\
0 & 1 \\
0 & 0
\end{bmatrix}
\quad
{\bf I}_{02+1} = 1/2
\begin{bmatrix}
1 & 0 \\
0 & 0 \\
0 & 1
\end{bmatrix}
\quad
{\bf I}_{12+0} = 1/2
\begin{bmatrix}
0 & 0 \\
1 & 0 \\
0 & 1
\end{bmatrix}
\label{chap01:eq04} \quad
\end{equation}
ce qui nous donne bien ${\bf J}^{-1}_{012} = {\bf I}_{01+2} {\bf J}^{-1}_{01} + 
{\bf I}_{02+1} {\bf J}^{-1}_{02} + {\bf I}_{12+0} {\bf J}^{-1}_{12}$.

Nous prenons d\`es lors le parti de n'accorder une pertinence physique qu'aux 
seules jacobiennes de robots concr\^ets.

Prenons l'exemple d'un robot dont les coordonn\'ees des points d'attaches sont 
${\bf A}_0 = (0.0, 0.0, 1.0)$, ${\bf A}_1 = (1.0, 0.0, 1.0)$, ${\bf A}_2 = 
(1.0, 1.0, 1.0)$ et ${\bf A}_3 = (0.0, 1.0, 1.0)$. Soit ${\bf S} = (0.2, 0.7, 
0.8)$ le point de d\'epart d'une trajectoire et ${\bf G} = (0.8, 0.6, 0.6)$ son 
point d'arriv\'ee. La projection horizontale de ${\bf S}$ appartient aux 
triangles dont les sommets respectifs sont ${\bf A}_0{\bf A}_1{\bf A}_3$ et 
${\bf A}_0{\bf A}_2{\bf A}_3$ ; la projection horizontale de ${\bf G}$ 
appartient quant \`a elle aux triangles dont les sommets respectifs sont ${\bf 
A}_1{\bf A}_2{\bf A}_3$ et ${\bf A}_0{\bf A}_1{\bf A}_2$. Supposons que pour 
${\bf S}$ nous partions de $CC_{013}$. Une premi\`ere solution visible 
dans \ref{chap01:fig0} pourrait \^etre de passer par une transition marginale 
correspondant \`a $CC_{13}$ de mani\`ere \`a continuer avec $CC_{123}$. Une 
autre possibilit\'e consiste \`a se mettre en configuration $CC_{023}$ 
(tranfert simple) puis d'utiliser un transfert simple de $CC_{023}$ vers 
$CC_{123}$. Une troisi\`eme enfin propose un seul transfert simple de 
$CC_{013}$ vers $CC_{012}$, mais ne peut \^etre parcourue en ligne droite. Deux 
questions se posent alors :
\begin{itemize}
 \item \`a quel point de la trajectoire op\'erer le transfert ?
  \item  deux transferts simples valent-ils mieux qu'un seul transfert marginal 
? qu'un seul transfert simple mais dont la trajectoire est plus complexe ? 
\end{itemize}

R\'epondre \`a ces questions n\'ecessite dans un premier temps de 
pouvoir comparer les propri\'et\'es locales des sous-robots associ\'es \`a 
chaque configuration de c\^ables impliqu\'ee, mais surtout de pouvoir 
pond\'erer les configurations et transferts de mani\`ere \`a d\'eterminer le 
trajet optimal. Nous proposons donc dans la suite de ce chapitre d'\'etudier 
plusieurs crit\`eres permettant une optimisation des trajets.

\section{Crit\`eres d'optimisation et d'\'evaluation}

\subsection{Travaux pr\'eliminaires}

Au contraire des CDPR pleinement contraints dont la pose est d\'etermin\'ee par 
les longueurs des c\^ables (quand un nombre suffisant de ceux-ci sont en 
tension), les robots en configuration suspendue utilise la gravit\'e pour 
contraindre la pose. Si ce choix de configuration poss\`ede plusieurs avantage, 
dont celui d'utiliser un nombre inf\'erieur de c\^ables (ce qui r\'eduit par 
exemple les possiblit\'es de collision), il rend cependant le syst\`eme plus 
sensibles aux perturbations, ce qui affecte la qualit\'e du mouvement du robot.

Le choix d'une configuration de c\^ables peut se faire lorsque plusieurs sont 
possibles pour une m\^eme pose. Nous partirons donc du postulat que pour une 
pose ${\bf X}$ il existe un ensemble $CC_\alpha$ de $M$ configurations $CC_{i}$ 
telles que $\forall i \in [1, N], {\bf X} \in El(CC_i)$. Certaines - voire 
toutes - de ces configurations supposent qu'un c\^able au moins est en 
tension nulle.

La strat\'egie g\'en\'eralement utilis\'e dans le contr\^ole des manipulateurs 
parall\`eles \`a c\^ables consiste \`a leur imposer une longueur la plus proche 
possible de la distance ${\bf A}_i{\bf B}$. Or, nous avons \'evoqu\'e \`a 
plusieurs reprises que l'incertitude sur les longueurs ne permettait pas de 
savoir quelle configuration allait succ\'eder au moment $t+1$ \`a celle prise 
effectivement par les c\^ables au moment $t$. Il est donc extr\^emement 
difficile - pour ne pas dire g\'en\'eralement impossible - d'avoir un 
contr\^ole sur les trajets (successions de configurations). Pour autant, 
l'analyse des sous-robots correspondants aux diff\'erentes configurations ne 
manquera pas de r\'ev'eler qu'\`a l'exception de cas rares, les propri\'et\'es 
du manipulateur ne sont pas \'equivalente pour la pose ${\bf X}$.

D\'es lors, plut\^ot que de tenter de garder les longueurs aussi proches 
que possible de leur valeurs th\'eoriques, nous avons propos\'e 
dans \cite{ramadour2014} de forcer la s\'el\'ection d'une configuration en 
ajoutant volontairement aux c\^ables qu'elle suppose mous une longueur 
suppl\'ementaire qui assure -- en fonction de l'incertitude sur les longueurs 
-- qu'il ne changera pas d'\'etat.

La premi\`ere \'etape consiste donc \`a d\'eterminer pour une pose donn\'ee 
l'ensemble des configurations de c\^ables dans lesquelles elle peut se 
retrouver. En hi\'erarchisant ensuite ces configurations \`a partir des 
propri\'et\'es des sous-robots, on en choisit une, puis on ajoute aux 
c\^ables qui n'interviennent pas dans la d\'efinition du sous-robot une 
longueur suppl\'ementaire tant que l'on ne souhaite pas modifier la 
configuration.

L'analyse -- lors d'une trajectoire -- en parall\`ele des propri\'et\'es des 
sous-robots associ\'ees aux configurations possibles permet de d\'eterminer les 
poses pour lesquelles il est souhaitable de changer de configuration. Au point 
de transfert d\'etermin\'e par la comparaison des crit\`eres respectifs, il 
suffit alors de redonner aux c\^ables impliqu\'ees dans la configuration 
post\'erieure leur longueur th\'eorique, et d'ajouter \`a ceux dont on ne veut 
plus se servir une longueur suppl\'ementaire.

Une fa\c con alternative de consid\'erer ce processus consiste \`a envisager 
que nous utilisons une flotte de robots coop\'erants les uns avec les autres 
lors d'une trajectoire. On comprend alors que le transfert d'un robot \`a un 
autre est un moment particulier du contr\^ole. Il implique entre-autres que la 
vitesse du d\'eplacement soit ralentie en ce point particulier de mani\`ere \`a 
minimiser l'influence des perturbations.

Si ce ralentissement peut \^etre consid\'er\'e comme un inconv\'enient, 
il offre toutefois deux avantages majeurs, \`a savoir :
\begin{itemize}
 \item l'existence pour une pose de plusieurs configurations de c\^ables 
possibles devient ici un atout, puisqu'il est permis de choisir celle qui 
optimisera un crit\`ere pr\'e\'etabli, alors que nous ne pouvions 
pr\'ec\'edemment que nous r\'ef\'erer au {\it pire des cas}
  \item en divisant la t\^ache en sous-robots distincts dont nous d\'ecidons 
lequel prend en charge le d\'eroulement d'un partie de la trajectoire, nous 
pouvons am\'eliorer les caract\'eristiques du robot global, d\'es lors que 
celles-ci ne sont plus minor\'ees par la situation la pire sur l'espace de 
travail, mais uniquement sur le minimum parmi les meilleures situations en 
chaque point de l'espace.
\end{itemize}

Nous allons \`a pr\'esent voir successivement comment nous proposons 
d'am\'eliorer la stabilit\'e puis la pr\'ecision du manipulateur.
 

\subsection{Am\'elioration de la stabilit\'e}

Am\'eliorer la stabilit\'e est primordial pour deux raisons :
\begin{itemize}
 \item lorsque le manipulateur est utilis\'e dans son contexte originel d'aide 
au d\'eplacement des personnes, cela permet d'une part d'assurer la 
s\'ecurit\'e du processus, en \'evitant la perte de contr\^ole d'un ou 
plusieurs degr\'es de libert\'e, d'autre part l'aisance de l'utilisation en 
n\'egociant les changements de configurations de c\^ables de mani\`ere souple
\item lors d'une op\'eration impliquant un asservissement visuel avec cam\'era 
embarqu\'ee, la moindre perturbation modifie la situation du r\'ef\'erentiel 
cam\'era par rapport au r\'ef\'erentiel de l'organe terminal. D\'es lors, la 
conversion des mesures effectu\'ees sur l'image en commande pour le 
manipulateur est fauss\'ee et l'asservissement \'echoue. Une perturbation trop 
forte peut \'egalement faire sortir la cible de l'image, ce qui conduit 
fatalement \`a l'impossiblit\'e de poursuivre l'utilisation de l'asservissement 
visuel.
\end{itemize} 


\subsubsection{Minimiser la tension}

Pour deux configurations donn\'ees, les tensions exerc\'ees sur les c\^ables de 
chacune d'entre elles diff\`ereront.

Prenons l'exemple du robot $4-1$ dont les points d'attache sont r\'epartis aux 
quatre coins les plus hauts d'un cube d'un m\`etre de c\^ot\'e. Fixons un point 
${\bf B}$ de coordonn\'ees $(0.20, 0.30, 0.80)^T$ (Fig.\ref{chap01:fig3}).

En ce point, deux configurations de c\^ables sont possibles : 
$\overline{CC}_{013}$(Fig.\ref{chap01:fig3view0}) et $\overline{CC}_{023}$(Fig.\ref{chap01:fig3view1}). Ces deux configurations 
permettent de contr\^oler les trois degr\'es de libert\'es en translation.

\begin{figure}[htp]
  \centering
 \subfloat[Le robot en configuration de câbles $\overline{CC}_{013}$]{\label{chap01:fig3view0}
 \def\svgwidth{.45\linewidth}
 \input{./chapter01/figures/tension02.pdf_tex}} \hfill
 \subfloat[Le robot en configuration de câbles $\overline{CC}_{023}$]{\label{chap01:fig3view1}
 \def\svgwidth{.45\linewidth}
 \input{./chapter01/figures/tension01.pdf_tex}} 
    \caption{\footnotesize{Les deux configurations de c\^ables possibles pour 
une pose ${\bf B} = (0.2, 0.3, 0.8)^T$}}
\label{chap01:fig3}
\end{figure}

Les jacobiennes inverses respectives sont :

\begin{equation}
{\bf J}^{-1}_{013} = 
\begin{bmatrix}
0.4851 & 0.7276 & -0.4851 \\
-0.9117 & 0.3419 & -0.2279 \\
0.2649 & -0.9272 & -0.2649
\end{bmatrix},
{\bf J}^{-1}_{023}
\begin{bmatrix}
0.4851 & 0.7276 & -0.4851 \\
-0.7396 & -0.6472 & -0.1849 \\
0.2649 & -0.9272 & -0.2649
\end{bmatrix}
\label{chap01:eq05}
\end{equation}

En posant $m = 1/g$, nous avons ${\boldmath {\mathcal F}} = (0, 0, -1)^T$, ce qui 
nous donne :

\begin{equation}
{\bf \tau}_{013} = 
\begin{bmatrix}
1.0308
0.8775  
1.1325
\end{bmatrix},
\quad
{\bf \tau}_{023}
\begin{bmatrix}
1.4431 \\
1.0817 \\
0.3775
\end{bmatrix}
\label{chap01:eq06}
\end{equation}

Nous pouvons d\'ej\`a v\'erifier qu'un passage non-contr\^ol\'e d'une 
configuration \`a l'autre produit des variations de tensions discontinues, 
alors m\^eme que les deux configurations ont en commun deux c\^ables sur trois.

De plus, la variance des tensions est plus importante pour la 
configuration $\overline{CC}_{023}$, ce qui dénote une répartition in\'egale des efforts sur les câbles.

Nous proposons donc de consid\'erer la norme des tensions th\'eoriques 
obtenues comme critère, et de choisir la configuration qui offre ainsi une meilleure 
r\'epartition, soit dans ce cas $\overline{CC}_{023}$.

Ainsi, en chaque pose, nous calculons pour toutes les configurations de câbles $\overline{CC}_i$ correspondant à une situation d'équilibre statique la mesure $\mathcal M_{\hbox{stab}_i}$ :

\begin{equation}
\mathcal M_{\hbox{stab}_i} = ||{\bf J}_i {\boldmath {\mathcal F}}||
\label{chap01:eq07}
\end{equation}

Posons à présent ${\bf B} = (0.3,0.3,0.3)^T$. Dans ce cas, les deux configurations de câbles possibles sont $\overline{CC}_{013}$ et $CC_{02}$.

Les jacobiennes inverses correspondantes sont :

\begin{equation}
{\bf J}^{-1}_{013} = 
\begin{bmatrix}
0.3665 & 0.3665 & -0.8552 \\  
-0.6767 & 0.2900 & -0.6767 \\  
0.2900 & -0.6767 & -0.6767
\end{bmatrix},
{\bf J}^{-1}_{02}
\begin{bmatrix}
0.3665 & 0.3665 & -0.8552 \\  
-0.5774 & -0.5774 & -0.5774
\end{bmatrix}
\label{chap01:eq08}
\end{equation}

En utilisant pour $A^{2,2}_{02}$ le sous-robot associé à la configuration de câbles $CC_{01}$ la pseudo-inverse de Moore-Penrose pour calculer les tensions théoriques correspondantes, nous obtenons :

\begin{equation}
{\bf \tau}_{013} = 
\begin{bmatrix}
0.4677 \\
0.4433 \\
0.4433
\end{bmatrix},
\quad
{\bf \tau}_{02}
\begin{bmatrix}
0.8185 \\ 
0.5196 
\end{bmatrix}
\label{chap01:eq09}
\end{equation}

Les mesures respectives sont $\mathcal M_{\hbox {stab}_{013}} = 0.7822$ et $\mathcal M_{\hbox {stab}_{02}} = 0.9695$, soit une sélection de la configuration $\overline{CC}_{013}$.

Toutes ces mesures pouvant être réalisées en amont, elles n'impactent pas le temps de calcul de la commande. Ainsi, pour notre robot $4-1$ cubique, pour une valeur de $Z = 0.5$ fixée, les figures \ref{} et \ref{} nous montrent respectivement quelle configuration de câbles sera privilégiée en tout point de l'espace de travail, et l'évolution du critère sur l'ensemble de celui-ci.

\subsubsection{Illustration}

\subsection{Am\'elioration de la pr\'ecision}


\subsubsection{Crit\`eres}

\subsubsection{Illustration}

\section{Algorithme de s\'election de configurations de c\^ables}


\subsection{L'analyse par intervalles}

\subsection{Algorithme}

\subsection{R\'esultats th\'eoriques}
