\chapter{Asservissement visuel d'un robot parall\`ele \`a c\^ables}

Dans un objectif de manipulation d'objets du quotidien dans des environnements 
dynamiques (lieux de vie des personnes) ou inconnus (lieux de catastrophes 
naturelles ou industrielles), la question se pose des divers avantages que 
peuvent offrir l'utilisation de la vision. Une cam\'era peut-\^etre utilis\'ee 
pour :
\begin{itemize}
  \item mesurer des propri\'et\'es articulaires (ex : longueur, direction de 
d\'epart des c\^ables)
  \item mesurer des propri\'et\'es op\'erationnelles (ex : pose de l'organe 
terminal)
  \item confirmer la pr\'esence d'un objet (ex : une mire pour \'etalonnage, 
une cible \`a saisir)
  \item mesurer les caract\'eristiques de cet objet (ex : aire, orientation, 
g\'eom\'etrie)
  \item $\cdots$
\end{itemize}

Afin de d\'etecter la pr\'esence d'un objet, on peut soit utiliser une ou 
plusieurs cam\'eras d\'eport\'ees couvrant l'ensemble de l'espace de travail du 
robot, soit une cam\'era embarqu\'ee. Or, ce qui d\'etermine souvent le choix 
d'utilisation d'un robot parall\`ele \`a c\^ables est la possibilit\'e 
d'\'evoluer dans un large espace de travail, qui n\'ecessiterait donc plusieurs 
cam\'eras de r\'esolution respectable. Au contraire, une cam\'era embarqu\'ee 
permet \`a elle seule de couvrir l'ensemble de l'espace, et ses limites 
\'eventuelles de r\'esolution peuvent \^etre contourn\'es par la diminution de 
la distance entre son support et sa cible. Lorsqu'il est question donc de 
manipulation dans en environnement dynamique pour un robot \`a c\^ables, le 
choix d'une cam\'era embarqu\'ee s'impose. La question se pose alors de 
l'utiliser seule ou de multiplier les dispositifs afin de multiplier les 
mesures. Le pr\'esent chapitre est consacr\'e \`a montrer qu'une seule cam\'era 
embarqu\'ee est suffisante pour obtenir la pr\'ecision requise aux op\'erations 
de manipulation et garantir une robustesse aux incertitudes diverses.

Pour cela, plusieurs d\'ecisions s'imposent dans l'\'elaboration du sch\'ema de 
contr\^ole utilis\'e, pr\'esent'ees dans une premi\`ere section. Nous nous 
attacherons dans une deuxi\`eme section \`a explorer plusieurs types 
d'informations visuelles \`a exploiter, pour finir dans une troisi\`eme section 
par pr'esenter les am\'elioration de pr\'ecision et robustesse qui sont ainsi 
permises.

\section{Elaboration du sch\'ema de contr\^ole}

\subsection{bla}

\subsection{bla}

\subsection{bla}


 