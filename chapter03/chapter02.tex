\chapter{Asservissement visuel d'un robot parall\`ele \`a c\^ables}

Dans un objectif de manipulation d'objets du quotidien dans des environnements 
dynamiques (lieux de vie des personnes) ou inconnus (lieux de catastrophes 
naturelles ou industrielles), la question se pose des divers avantages que 
peuvent offrir l'utilisation de la vision. Une cam\'era peut-\^etre utilis\'ee 
pour :
\begin{itemize}
  \item mesurer des propri\'et\'es articulaires (ex : longueur, direction de 
d\'epart des c\^ables)
  \item mesurer des propri\'et\'es op\'erationnelles (ex : pose de l'organe 
terminal)
  \item confirmer la pr\'esence d'un objet (ex : une mire pour \'etalonnage, 
une cible \`a saisir)
  \item mesurer les caract\'eristiques de cet objet (ex : aire, orientation, 
g\'eom\'etrie)
  \item $\cdots$
\end{itemize}

Afin de d\'etecter la pr\'esence d'un objet, on peut soit utiliser une ou 
plusieurs cam\'eras d\'eport\'ees couvrant l'ensemble de l'espace de travail du 
robot, soit une cam\'era embarqu\'ee. Or, ce qui d\'etermine souvent le choix 
d'utilisation d'un robot parall\`ele \`a c\^ables est la possibilit\'e 
d'\'evoluer dans un large espace de travail, qui n\'ecessiterait dans le 
cas d'un choix de cam\'eras d\'eport\'ees plusieurs dispositifs de r\'esolution 
respectable. Au contraire, une cam\'era embarqu\'ee  permet \`a elle seule de 
couvrir l'ensemble de l'espace, et ses limites  \'eventuelles de r\'esolution 
peuvent \^etre contourn\'es par la diminution de la distance entre son support 
et sa cible. Lorsqu'il est question donc de manipulation dans en environnement 
dynamique pour un robot \`a c\^ables, le choix d'une cam\'era embarqu\'ee 
s'impose. La question se pose alors de l'utiliser seule ou de multiplier les 
dispositifs afin de multiplier les mesures. Le pr\'esent chapitre est consacr\'e 
\`a montrer qu'une seule cam\'era embarqu\'ee est suffisante pour obtenir la 
pr\'ecision requise aux op\'erations de manipulation et garantir une robustesse 
aux incertitudes diverses.

Pour cela, plusieurs d\'ecisions s'imposent dans l'\'elaboration du sch\'ema de 
contr\^ole utilis\'e, pr\'esent'ees dans une premi\`ere section. Nous nous 
attacherons dans une deuxi\`eme section \`a explorer plusieurs types 
d'informations visuelles \`a exploiter, pour finir dans une troisi\`eme section 
par pr'esenter les am\'elioration de pr\'ecision et robustesse qui sont ainsi 
permises.

\section{Elaboration du sch\'ema de contr\^ole}

\section{Primitives et matrices d'interaction associ\'ees}

Nous nous attachons dans cette section \`a pr\'esenter plusieurs informations 
visuelles exploitables dans notre situation. Le suivi de points que nous 
analyserons en premier peut \^etre utilis\'e par exemple lorsque nous suivons 
une mire ou lorsque nous exploitons les points d'int\'er\^et d'un objet. Les 
droites sont des primitives visuelles qui ont par exemple \'et\'e utilis\'ees 
dans le cadre de l'asservissement des jambes rigides des manipulateurs 
parall\`eles ou pour la d\'etermination des directions de sortie des c\^ables 
dans les travaux cit\'es en introduction. Ce type d'information peut 
\'egalement \^etre utilis\'e lorsque la g\'eom\'etrie de l'objet le permet, en 
compl\'ement d'autres primitives. Enfin, nous nous attacherons 
particuli\`erement \`a l'utilisation des moments d'un objet dans une image, 
tels que l'aire ou le centre de gravit\'e de la cible. Ce dernier type de 
primitive permet d'obtenir une loi de commande simplifi\'ee, tout en restant 
pertinent pour les applications vis\'ees par le contexte dans lequel s'est 
d\'eroul\'e ce travail.

\subsubsection{Le point}


\subsection{bla}

\subsection{bla}


 